\documentclass{article}
\usepackage{amsthm}
\usepackage{amsmath}
\usepackage{amsfonts}

\newtheorem*{definition}{Definição}
\newtheorem*{theorem}{Teorema}
\newtheorem*{exemple}{Exemplo}
\newtheorem*{atencao}{Atenção}
\newtheorem*{proposicao}{Proposição}
\usepackage{mathtools}
\usepackage{xfrac}
\DeclarePairedDelimiter\abs{\lvert}{\rvert}%
\DeclareMathOperator{\Ima}{Im}

\setlength{\textwidth}{450pt}
\setlength{\marginparwidth}{0pt}
\setlength{\marginparsep}{0pt}
\usepackage[left=2.5cm, right=2.5cm]{geometry}

\title{Conjuntos Abertos \\ \large Introdução a Topologia}
\author{Yuri Kosfeld}
\date{Abril 2025}


\begin{document}

\maketitle

\section*{Bolas Abertas}

O estudo de conjuntos abertos é motivado por querer entender a forma de um espaço metrico olhando 
para as vizinhanças de todos os pontos desse espaço. Vamos relembrar alguns
detalhes importantes sobre distâncias.

\begin{definition}[\textbf{Distância}]
    Seja M um conjunto. Uma \textbf{distância} em M é uma função $d: M \times M \rightarrow \left[0, \infty \right)$
    que satisfaz as seguintes propriedades: 
    \begin{enumerate}
        \item $d(x,x) = 0$ para todo $x \in M$.
        \item $d(x,y) > 0$ para todos $x, y \in M$ com $x \neq y$.
        \item $d(x, y) = d(y, x)$ para todos $x,y \in M$.
        \item $d(x,z) \leq d(x, y) + d(y, z)$ para todos $x, y, z \in M$.
    \end{enumerate}
\end{definition}

Conseguimos agora então entender a proximidade de dois pontos. Para então formalizar matematicamente
essa ideia de vizinhança de um ponto, vamos definir o que é uma \textbf{bola aberta}.

\begin{definition}[\textbf{Bola Aberta}]
    Sejam (M, d) um espaço métrico, $x\in M$ e $\varepsilon > 0$. Definimos a \textbf{bola aberta}
    centrada em x e de raio $\varepsilon$ como:
    \[ B(x, \varepsilon) = \{ y \in M \: | \: d(x, y) < \varepsilon \} \]
\end{definition}

Não é dificil notar que em $\mathbb{R}$ com a distância usual, temos que a bola $B(x, \varepsilon) = (x - \varepsilon, x + \varepsilon)$.
Pela definição, segue que $B(x, \varepsilon) = \{ y \in \mathbb{R} \: | \: | x - y | < \varepsilon \}$.
Então tomando $y \in B(x, \varepsilon)$ temos $| x - y | < \varepsilon$ e então
\begin{align*}
    -\varepsilon < x - y < \varepsilon \\
    -\varepsilon < y - x < \varepsilon \\
    x - \varepsilon < y < x + \varepsilon\\
    \Leftrightarrow  y \in (x - \varepsilon, x + \varepsilon)
\end{align*}

Outro exemplo é a bola em $\mathbb{R}^2$, com a distância euclidiana centrada na origem:
\[B((0, 0), \varepsilon) = \{ (x,y) \in \mathbb{R}^2 \: | \: \sqrt{x^2 + y^2} < \varepsilon\} = \{ (x,y) \in \mathbb{R}^2 \: | \: x^2 + y^2 < \varepsilon^2\}\]
Assim note, que com essa distância, as bolas em $\mathbb{R}^2$ são discos.

Um resultado interessante é o que acontecem com bolas no espaço produto levando em conta a distância produto.
Sejam $(M_1, d_1), \dots ,(M_n, d_n)$ espaços metrico, e defina $M = M_1 \times \dots \times M_n $ com a distância produto.
Vamos mostrar que a bola produto é o produto das bolas, ou seja, $B((x_1, \dots, x_n), \varepsilon) = B(x_1, \varepsilon) \times \dots \times B(x_n, \varepsilon)$.
Lembre que a distância produto é dada por 
\[d_{max}(x, y) = \max\{d_1(x_1, y_1), \dots, d_n(x_n, y_n)\}\]
Então segue que se $y \in B((x_1, \dots, x_n), \varepsilon)$, $d_i(x_i, y_i) < \varepsilon \quad \forall i$.
Equivalente a dizer que $\forall i \quad y_i \in B(x_i, \varepsilon)$ e então $y \in  B(x_1, \varepsilon) \times \dots \times  B(x_n, \varepsilon)$.
Agora tome $y \in  B(x_1, \varepsilon) \times \dots \times  B(x_n, \varepsilon)$, note que para cada $i$, $y_i \in B(x_i, \varepsilon)$.
Logo pela definição de bola e da distância produto vale $d(y, x) < \varepsilon$. Donde segue que $y \in B(x, \varepsilon)$.

Podemos então perceber que a bola com a distância produto são quadrados de lado $2 \varepsilon$.

Outra bola interessante é a bola da distância induzida. Seja então (M, d) um espaço metrico e $A \subset M$ 
com a distância induzida. Vamos representar a bola da distância induzida em A por $B_A(x, \varepsilon)$,
para um dado $x \in A$ e raio $\varepsilon$. Vamos provar que $B_A(x, \varepsilon) = B(x, \varepsilon) \cap A$.
Tome $y \in B_A(x, \varepsilon)$, então $d(x, y) < \varepsilon$, donde segue que $y \in B(x, \varepsilon) \cap A$.
Agora tome $y \in B(x, \varepsilon) \cap A$, então $d_A(x, y) < \varepsilon$ e segue que $y \in B_A(x, \varepsilon)$.

\section*{Conjuntos Abertos}

Primeiro vamos definir alguns objetos para depois entender as suas propriedades e relações.

\begin{definition}[\textbf{Conjunto Aberto}]
    $U \subset M$ é \textbf{conjunto aberto} se $\forall \: x \in U$ $\exists \: \varepsilon > 0$ tal que $B(x, \varepsilon) \subset U$.
\end{definition}

\begin{definition}[\textbf{Vizinhança}]
    U é uma \textbf{vizinhança} de x se U aberto tal que $x \in U$.
\end{definition}

\begin{definition}[\textbf{Topologia}]
    Seja (M, d) espaço metrico, dizemos que \textbf{topologia} é a familia de todos os subconjuntos abertos de M.
    \[ \mathcal{T} = \{ U \subset M \: | \: \text{U é aberto}\} \]
\end{definition}

Um primeiro exemplo simples é que todo intervalo aberto (a,b) em $\mathbb{R}$ é aberto.
Para isso basta tomar $\varepsilon \leq \min\{x-a. b-x\}$. Como já vimos anteriormente, 
$B(x, \varepsilon) = (x - \varepsilon, x + \varepsilon)$, então se $y \in (x - \varepsilon, x + \varepsilon)$
temos $ x - \varepsilon < y < x + \varepsilon $, donde segue:
\begin{align*}
    y &> x - \varepsilon \\
    y &> x - x + a = a
\end{align*}
e
\begin{align*}
    y &< x + \varepsilon \\
    y &< x - x + b = b
\end{align*}
Logo $y \in (a, b)$.

Podemos também mostrar que em $\mathbb{R}$, qualquer conjunto formado por apenas um ponto, 
não é aberto em $\mathbb{R}$. Vamos mostrar que {x} não é aberto em $\mathbb{R}$. Note que
para todo $\varepsilon > 0$, $(x - \varepsilon, x + \varepsilon) \setminus \{x\} \neq \emptyset$.
Tome então $y \in (x - \varepsilon, x + \varepsilon) \setminus \{x\}$, logo $y \neq x$ e então 
$y \notin \{x\}$. Assim $B(x, \varepsilon) \subseteq \{x\}$ e não é aberto.

Podemos extender nosso exemplo para o caso de $\mathbb{R}^2$. Como vimos, um intervalo aberto de $\mathbb{R}$
é aberto em $\mathbb{R}$. Tomando então dois invervalos abertos $(a, b), (c, d)$ temos 
$\varepsilon_1, \varepsilon_2 > 0$ tais que $B(x, \varepsilon_1) \subset (a, b)$ e $B(y, \varepsilon_2) \subset (c, d)$.
Defina então $\varepsilon = \min\{\varepsilon_1, \varepsilon_2\}$ e então, lembrando da propriedade de que uma bola aberta em $\mathbb{R}^2$ é igual ao produto de duas bolas abertas em $\mathbb{R}$, segue:
\[B((x, y), \varepsilon) = (x-\varepsilon, x+\varepsilon) \times (y-\varepsilon, y+\varepsilon) \subset (x-\varepsilon_1, x+\varepsilon_1) \times (y-\varepsilon_2, y+\varepsilon_2) \subset (a, b) \times (c, d)\]
Logo é aberto em $\mathbb{R}^2$.

Vamos mostrar agora que toda bola aberta é um subconjunto aberto.
Tome então $z \in B(x, \varepsilon)$. Precisamos achar um $\varepsilon' >0$ tal que $B(z, \varepsilon') \subset B(x, \varepsilon)$.
Vamos mostar que $\varepsilon' = \varepsilon - d(x, z)$ é suficiente.
Tome $y \in B(z, \varepsilon')$, pela desigualdade triangular temos:
\[d(y, x) \leq d(y, z) + d(z, x) < \varepsilon' + d(z, x) = \varepsilon - d(z, x) + d(z, x) = \varepsilon\]
Logo $y \in B(x, \varepsilon)$. E então a bola aberta é um conjunto aberto.

\end{document}