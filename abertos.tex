\documentclass{article}
\usepackage{amsthm}
\usepackage{amsmath}
\usepackage{amsfonts}

\newtheorem*{definition}{Definição}
\newtheorem*{theorem}{Teorema}
\newtheorem*{exemple}{Exemplo}
\newtheorem*{atencao}{Atenção}
\newtheorem*{proposicao}{Proposição}
\usepackage{mathtools}
\usepackage{xfrac}
\DeclarePairedDelimiter\abs{\lvert}{\rvert}%
\DeclareMathOperator{\Ima}{Im}

\setlength{\textwidth}{450pt}
\setlength{\marginparwidth}{0pt}
\setlength{\marginparsep}{0pt}
\usepackage[left=2.5cm, right=2.5cm]{geometry}

\title{Conjuntos Abertos \\ \large Introdução a Topologia}
\author{Yuri Kosfeld}
\date{Abril 2025}


\begin{document}

\maketitle

O estudo de conjuntos abertos é motivado por querer entender a forma de um espaço metrico olhando 
para as vizinhanças de todos os pontos desse espaço. Vamos relembrar alguns
detalhes importantes sobre distâncias.

\begin{definition}[\textbf{Distância}]
    Seja M um conjunto. Uma \textbf{distância} em M é uma função $d: M \times M \rightarrow \left[0, \infty \right)$
    que satisfaz as seguintes propriedades: 
    \begin{enumerate}
        \item $d(x,x) = 0$ para todo $x \in M$.
        \item $d(x,y) > 0$ para todos $x, y \in M$ com $x \neq y$.
        \item $d(x, y) = d(y, x)$ para todos $x,y \in M$.
        \item $d(x,z) \leq d(x, y) + d(y, z)$ para todos $x, y, z \in M$.
    \end{enumerate}
\end{definition}

Conseguimos agora então entender a proximidade de dois pontos. Para então formalizar matematicamente
essa ideia de vizinhança de um ponto, vamos definir o que é uma \textbf{bola aberta}.

\begin{definition}[\textbf{Bola Aberta}]
    Sejam (M, d) um espaço métrico, $x\in M$ e $\varepsilon > 0$. Definimos a \textbf{bola aberta}
    centrada em x e de raio $\varepsilon$ como:
    \[ B(x, \varepsilon) = \{ y \in M \: | \: d(x, y) < \varepsilon \} \]
\end{definition}

Não é dificil notar que em $\mathbb{R}$ com a distância usual, temos que a bola $B(x, \varepsilon) = (x - \varepsilon, x + \varepsilon)$.
Pela definição, segue que $B(x, \varepsilon) = \{ y \in \mathbb{R} \: | \: | x - y | < \varepsilon \}$.
Então tomando $y \in B(x, \varepsilon)$ temos $| x - y | < \varepsilon$ e então
\begin{align*}
    -\varepsilon < x - y < \varepsilon \\
    -\varepsilon < y - x < \varepsilon \\
    x - \varepsilon < y < x + \varepsilon\\
    \Leftrightarrow  y \in (x - \varepsilon, x + \varepsilon)
\end{align*}

Outro exemplo é a bola em $\mathbb{R}^2$, com a distância euclidiana centrada na origem:
\[B((0, 0), \varepsilon) = \{ (x,y) \in \mathbb{R}^2 \: | \: \sqrt{x^2 + y^2} < \varepsilon\} = \{ (x,y) \in \mathbb{R}^2 \: | \: x^2 + y^2 < \varepsilon^2\}\]
Assim note, que com essa distância, as bolas em $\mathbb{R}^2$ são discos.

Um resultado interessante é o que acontecem com bolas no espaço produto levando em conta a distância produto.
Sejam $(M_1, d_1), \dots ,(M_n, d_n)$ espaços metrico, e defina $M = M_1 \times \dots \times M_n $ com a distância produto.
Vamos mostrar que $B((x_1, \dots, x_n), \varepsilon) = B(x_1, \varepsilon) \times \dots \times B(x_n, \varepsilon)$.
Lembre que a distância produto é dada por 
\[d_{max}(x, y) = \max\{d_1(x_1, y_1), \dots, d_n(x_n, y_n)\}\]
Então segue que se $y \in B((x_1, \dots, x_n), \varepsilon)$, $d_i(x_i, y_i) < \varepsilon \quad \forall i$.
Equivalente a dizer que $\forall i \quad y_i \in B(x_i, \varepsilon)$ e então $y \in  B(x_1, \varepsilon) \times \dots \times  B(x_n, \varepsilon)$.
Agora tome $y \in  B(x_1, \varepsilon) \times \dots \times  B(x_n, \varepsilon)$. 

\end{document}