\documentclass{article}
\usepackage{amsthm}
\usepackage{amsmath}
\usepackage{amsfonts}

\newtheorem*{definition}{Definição}
\newtheorem*{theorem}{Teorema}
\newtheorem*{exemple}{Exemplo}
\newtheorem*{atencao}{Atenção}
\newtheorem*{proposicao}{Proposição}
\usepackage{mathtools}
\usepackage{xfrac}
\DeclarePairedDelimiter\abs{\lvert}{\rvert}%
\DeclareMathOperator{\Ima}{Im}

\setlength{\textwidth}{450pt}
\setlength{\marginparwidth}{0pt}
\setlength{\marginparsep}{0pt}
\usepackage[left=2.5cm, right=2.5cm]{geometry}

\title{Abertos \\ \large Introdução a Topologia}
\author{Yuri Kosfeld}
\date{Abril 2025}


\begin{document}

\maketitle

\begin{definition}
    Seja M um conjunto. Uma \textbf{distância} em M é uma função $d: M \times M \rightarrow \left[0, \infty \right)$
    que satisfaz as seguintes propriedades: 
    \begin{enumerate}
        \item $d(x,x) = 0$ para todo $x \in M$.
        \item $d(x,y) > 0$ para todos $x, y \in M$ com $x \neq y$.
        \item $d(x, y) = d(y, x)$ para todos $x,y \in M$.
        \item $d(x,z) \leq d(x, y) + d(y, z)$ para todos $x, y, z \in M$.
    \end{enumerate}
\end{definition}

\end{document}