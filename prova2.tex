\documentclass{article}
\usepackage{amsthm}
\usepackage{amsmath}
\usepackage{amsfonts}

\newtheorem*{definition}{Definição}
\newtheorem*{theorem}{Teorema}
\newtheorem{corolario}{Corolário}
\newtheorem*{exemplo}{Exemplo}
\newtheorem*{atencao}{Atenção}
\newtheorem*{proposicao}{Proposição}
\newtheorem*{lema}{Lema}
\newtheorem*{prova}{Demonstração}
\newcommand{\R}{\mathbb{R}}
\newcommand{\eps}{\varepsilon}

\usepackage{mathtools}
\usepackage{xfrac}
\DeclarePairedDelimiter\abs{\lvert}{\rvert}%
\DeclareMathOperator{\Ima}{Im}

\setlength{\textwidth}{450pt}
\setlength{\marginparwidth}{0pt}
\setlength{\marginparsep}{0pt}
\usepackage[left=2.5cm, right=2.5cm]{geometry}

\title{Definições, Propriedades e Resultados P2 \\ \large Introdução a Topologia}
\author{Yuri Kosfeld}
\date{Abril 2025}

\begin{document}

\maketitle

\section{Variedades}

\begin{definition}[\textbf{Variedade e Atlas}]
    Dado M um espaço métrico, dizemos que M é uma \textbf{variedade de dimensão n}
    se, existe uma família de funções $A = \{ \varphi_i: U_i \subset M \rightarrow V_i \subset \R^n\}$
    que satisfazem:
    \begin{enumerate}
        \item $\forall i \in I$, $U_i$ é aberto em M e $V_i$ é aberto em $\R^n$.
        \item $\bigcup_{i \in I} U_i = M$.
        \item $\forall i \in I$, $\varphi_i$ é um homeomorfismo.
    \end{enumerate}
    Denotaremos por $M^n$.
    Definimos também A como o \textbf{atlas da variedade}.
\end{definition}

\begin{exemplo}
    \
    \begin{itemize}
        \item $\R^n$ é uma variedade
        \item qualquer aberto de $\R^n$ é uma variedade
        \item $S^1$ é uma variedade
    \end{itemize}
\end{exemplo}

\begin{definition}[\textbf{Fibrado}]
    Sejam M, N e F espaços métricos. M é um \textbf{fibrado com base N e fibras F} se existem:
    \begin{enumerate}
        \item Uma aplicação $\pi: M \rightarrow N$ continua tal que $\forall x \in N$, 
            a fibra $\pi^{-1}(x)$ é homeomorfa a F.
        \item Uma família de abertos $\{U_i\}_{i \in I}$ de N tal que 
            \[\bigcup_{i \in I} U_i = N\]
        \item Para cada $i \in I$, existe um homeomorfismo $\varphi_i: U_i \times F \rightarrow \pi^{-1}(U_i) \subset M$
            satisfazendo 
            \[\pi(\varphi(x, v)) = x \quad \forall x \in U_i, v \in F\]
    \end{enumerate}
\end{definition}

\section{Bases}

\begin{definition}[\textbf{Base}]
    Dado M um espaço métrico. Uma \textbf{base} de M é uma coleção de abertos
    $\beta = \{\beta_i\}_{i \in I}$ que verifica:
    $\forall U \subset M$ aberto, $\exists I' \subset I$ tal que 
    \[U = \bigcup_{i \in I'} \beta _i\] 
\end{definition}

\begin{exemplo}
    $\beta = \{B(x, r) \: | \: x \in M, r > 0\}$ é uma base.
\end{exemplo}

\begin{lema}
    Seja M espaço métrico e $\beta = \{\beta_i\}_{i \in I}$ uma coleção de abertos.
    Se $\beta$ satisfaz: $\forall U \subset M$ aberto e $\forall x \in U$, $\exists i \in I$ 
    tal que $x \in B_i \subset U$, então $\beta$ é uma base de M.
\end{lema}

\begin{prova}
    Seja U aberto de M. 
    Queremos mostrar que U é união de elementos de $\beta$.
    Por hipotese: $\forall x \in U$, $\exists i(x) \in I$ tal que $x \in B_{i(x)} \subset U$.
    Logo 
    \[U = \bigcup_{x \in U} x \subset \bigcup_{x \in U} B_{i(x)} \subset \bigcup_{x \in U} U = U\]
    \[\Rightarrow U = \bigcup_{x \in U} B_{i(x)}\]
     e assim $\beta$ é base.
\end{prova}

\begin{atencao}
    As bases não são únicas!
\end{atencao}

\begin{definition}[\textbf{Base Enumerável}]
    Um espaço métrico M admite \textbf{base enumerável} se existe $\beta = \{\beta_i\}_{i \in I}$ base
    tal que I é enumerável. 
\end{definition}

\begin{exemplo}
    $\R$ admite base enumerável: $\beta = \{ (a, b) \subset \R \: | \: a, b \in \mathbb{Q} \}$.
    Sabemos que $\mathbb{Q}$ é enumerável e $\beta$ é base pelo \textbf{lema} anterior:
    dado $U \in \R$ e $x \in U$, temos que $x \in (a, b) \subset B(x, \eps) \subset U$.
\end{exemplo}

\begin{atencao}
    O produto de espaços que admitem base enumerável também admite base enumerável.
\end{atencao}

\begin{proposicao}
    Seja M espaço métrico. São equivalentes:
    \begin{enumerate}
        \item M admite base enumerável.
        \item $\exists D \subset M$ enumerável e denso (M é separavél).
    \end{enumerate}
\end{proposicao}

\begin{prova}
    $(1) \Rightarrow (2)$: \\
    Seja $\beta = \{ \beta_k\}$ base enumerável. 
    Para cada $k \in \mathbb{N}$ escolhemos $x_k \in \beta_k$,
    e então defina $D = \{x_k \: | \: k \in \mathbb{N}\}$. 
    D é natualmente enumerável pela construção, então precisamos verificar ainda que D é denso.
    Para ver que D é denso, basta ver que dado U aberto, vale $U \cap D \neq \emptyset$.
    Dado U, $\exists \: k$ tal que $\beta_k \neq \emptyset$ e $\beta_k \subset U$.
    Logo $x_k \in D \cap U$, e portanto D é denso.\\
    $(2) \Rightarrow (1)$: \\
    Seja D denso e enumerável e considere
    \[\beta = \{B(y, r) \: | \: y \in D, r \in \mathbb{Q}^+\}\]
    Temos que $\beta$ é enumerável e é base pelo $\textbf{lema}$:\\
    Seja U aberto e $x \in U$. Queremos B elemento da base tal que $x \in B \subset U$.
    Temos $\exists \: y \in D \cap B(x, \eps / 2)$, ou seja, $d(x, y) < \eps / 2$.
    Temos também, $\exists \: r \in \mathbb{Q}$ tal que $d(x, y) < r < \eps / 2$.
    Para este r, vale que $x \in B(y, r)$. 
    Assim $B(y, r) \subset B(x, \eps) \subset U$.
\end{prova}

\begin{definition}[\textbf{Base Local}]
    Dado M espaço métrico e $x \in M$, uma \textbf{base local de M em x} é $\beta = \{\beta_i\}_{i \in I}$
    de vizinhanças de x que verifica:
    $\forall \: U$ vizinhança de x, $\exists \: \beta_i \in \beta$ tal que $x \in B_i \subset U$.
\end{definition}

\begin{definition}[\textbf{Primeiro Axioma de Enumerabilidade}]
    Todo ponto admite base local enumerável.
\end{definition}

\begin{atencao}
    Todo espaço métrico satisfaz o Primeiro Axioma de Enumerabilidade. Basta tomar:
    \[\beta = \{ B(x, 1/k) \}\]
\end{atencao}

\section{Conexidade}

\begin{definition}[\textbf{Separação}]
    Seja M espaço métrico Uma \textbf{separação} de M é um par de subconjuntos de M $\{ A, B\}$ que verificam:
    \begin{itemize}
        \item $A \cup B = M$.
        \item $A \cap B = \emptyset$.
        \item A e B são abertos.
    \end{itemize}
    Chamamos $\{M, \emptyset\}$ de \textbf{separação trivial}.
\end{definition}

\begin{definition}[\textbf{Conexo}]
    M é \textbf{conexo} se a única separação que admite é a trivial. Se M não é conexo, dizemos que M é \textbf{desconexo}.
\end{definition}

\begin{atencao}
    Se $\{A, B\}$ é separação, então A e B são fechados.
\end{atencao}

\begin{proposicao}
    M é conexo se e somente se os únicos conjuntos abertos e fechados de M são M e $\emptyset$.
\end{proposicao}

\begin{prova}
    COMPLETAR!
\end{prova}

\begin{atencao}
    $X \subset M$ é conexo se X com a topologia relativa é conexo. 
\end{atencao}

\begin{proposicao}
    Se $f: M \rightarrow N$ é contínua e M é conexo, então $f(M)$ também é conexo.
\end{proposicao}

\begin{prova}
    Queremos ver que f(M) é conexo. Seja $\{A, B\}$ separação de f(M).
    $\exists \: A', B'$ abertos em N tais que 
    \[A' \cap f(M) = A\]
    \[B' \cap f(M) = B\]
    Como f é contínua, $f^{-1}(A')$ e $f^{-1}(B')$ são abertos de M.
    Logo
    \begin{align*}
        f^{-1}(A') &= f^{-1}(f(M) \cap A') \\
                   &= f^{1}(A) 
    \end{align*}
    e também $f^{-1}(B') = f^{-1}(B)$.\\
    Vamos mostrar que $\{f^{-1}(A), f^{-1}(B)\}$ é separação de M.
    \begin{itemize}
        \item $f^{-1}(A)$ e $f^{-1}(B)$ são abertos.
        \item $M = f^{-1}(f(M)) = f^{-1}(A \cup B) = f^{-1}(A) \cup f^{-1}(B)$.
        \item $f^{-1}(A) \cap f^{-1}(B) = f^{-1}(A \cap B) = \emptyset$.
    \end{itemize}
    Como M é conexo, temos duas opções: $f^{-1}(A) = \emptyset$ ou $f^{-1}(B) = \emptyset$.
    Se $f^{-1}(A) = \emptyset$, como $A \subset Im(f)$, segue que $A = \emptyset$ e então $B = f(M)$.
    Se $f^{-1}(B) = \emptyset$, analogo ao caso anterior temos, $B = \emptyset$ e $A = f(M)$.\\
    Assim $\{A, B\}$ é a separação trivial de f(M).
\end{prova}

\begin{corolario}
    Seja $f: M \rightarrow N$ homeomorfismo. Então, M é conexo se e somente se N é conexo.
\end{corolario}

\begin{exemplo}
    \
    \begin{itemize}
        \item $M = \{x\}$ é conexo.
        \item $[0, 1) \cup (1, 2]$ não é conexo. Basta perceber que $\{[0, 1), (1, 2]\}$ é separação, já que
            $[0, 1) = (-1, 1) \cap M$ e $(1, 2] = (1, 3) \cap M$ são abertos. 
    \end{itemize}
\end{exemplo}

\begin{theorem}
    Se $I \subset \R$ é um intervalo, então I é conexo. Mais ainda, os únicos conexos de $\R$ são intervalos ou conjuntos com um único ponto.
\end{theorem}

\begin{prova}
    COMPLETAR!
\end{prova}

\end{document}