\documentclass{article}
\usepackage{amsthm}
\usepackage{amsmath}
\usepackage{amsfonts}

\newtheorem*{definition}{Definição}
\newtheorem*{theorem}{Teorema}
\newtheorem{corolario}{Corolário}
\newtheorem*{exemplo}{Exemplo}
\newtheorem*{atencao}{Atenção}
\newtheorem*{proposicao}{Proposição}
\newtheorem*{lema}{Lema}
\newtheorem*{prova}{Demonstração}
\newcommand{\R}{\mathbb{R}}
\newcommand{\eps}{\varepsilon}

\usepackage{mathtools}
\usepackage{xfrac}
\DeclarePairedDelimiter\abs{\lvert}{\rvert}%
\DeclareMathOperator{\Ima}{Im}

\setlength{\textwidth}{450pt}
\setlength{\marginparwidth}{0pt}
\setlength{\marginparsep}{0pt}
\usepackage[left=2.5cm, right=2.5cm]{geometry}

\title{Definições, Propriedades e Resultados P2 \\ \large Introdução a Topologia}
\author{Yuri Kosfeld}
\date{Abril 2025}

\begin{document}

\maketitle

\section{Variedades}

\begin{definition}[\textbf{Variedade e Atlas}]
    Dado M um espaço métrico, dizemos que M é uma \textbf{variedade de dimensão n}
    se, existe uma família de funções $A = \{ \varphi_i: U_i \subset M \rightarrow V_i \subset \R^n\}$
    que satisfazem:
    \begin{enumerate}
        \item $\forall i \in I$, $U_i$ é aberto em M e $V_i$ é aberto em $\R^n$.
        \item $\bigcup_{i \in I} U_i = M$.
        \item $\forall i \in I$, $\varphi_i$ é um homeomorfismo.
    \end{enumerate}
    Denotaremos por $M^n$.
    Definimos também A como o \textbf{atlas da variedade}.
\end{definition}

\begin{exemplo}
    \
    \begin{itemize}
        \item $\R^n$ é uma variedade
        \item qualquer aberto de $\R^n$ é uma variedade
        \item $S^1$ é uma variedade
    \end{itemize}
\end{exemplo}

\begin{definition}[\textbf{Fibrado}]
    Sejam M, N e F espaços métricos. M é um \textbf{fibrado com base N e fibras F} se existem:
    \begin{enumerate}
        \item Uma aplicação $\pi: M \rightarrow N$ continua tal que $\forall x \in N$, 
            a fibra $\pi^{-1}(x)$ é homeomorfa a F.
        \item Uma família de abertos $\{U_i\}_{i \in I}$ de N tal que 
            \[\bigcup_{i \in I} U_i = N\]
        \item Para cada $i \in I$, existe um homeomorfismo $\varphi_i: U_i \times F \rightarrow \pi^{-1}(U_i) \subset M$
            satisfazendo 
            \[\pi(\varphi(x, v)) = x \quad \forall x \in U_i, v \in F\]
    \end{enumerate}
\end{definition}

\section{Bases}

\begin{definition}[\textbf{Base}]
    Dado M um espaço métrico. Uma \textbf{base} de M é uma coleção de abertos
    $\beta = \{\beta_i\}_{i \in I}$ que verifica:
    $\forall U \subset M$ aberto, $\exists I' \subset I$ tal que 
    \[U = \bigcup_{i \in I'} \beta _i\] 
\end{definition}

\begin{exemplo}
    $\beta = \{B(x, r) \: | \: x \in M, r > 0\}$ é uma base.
\end{exemplo}

\begin{lema}
    Seja M espaço métrico e $\beta = \{\beta_i\}_{i \in I}$ uma coleção de abertos.
    Se $\beta$ satisfaz: $\forall U \subset M$ aberto e $\forall x \in U$, $\exists i \in I$ 
    tal que $x \in B_i \subset U$, então $\beta$ é uma base de M.
\end{lema}

\begin{prova}
    Seja U aberto de M. 
    Queremos mostrar que U é união de elementos de $\beta$.
    Por hipotese: $\forall x \in U$, $\exists i(x) \in I$ tal que $x \in B_{i(x)} \subset U$.
    Logo 
    \[U = \bigcup_{x \in U} x \subset \bigcup_{x \in U} B_{i(x)} \subset \bigcup_{x \in U} U = U\]
    \[\Rightarrow U = \bigcup_{x \in U} B_{i(x)}\]
     e assim $\beta$ é base.
\end{prova}

\begin{atencao}
    As bases não são únicas!
\end{atencao}

\begin{definition}[\textbf{Base Enumerável}]
    Um espaço métrico M admite \textbf{base enumerável} se existe $\beta = \{\beta_i\}_{i \in I}$ base
    tal que I é enumerável. 
\end{definition}

\begin{exemplo}
    $\R$ admite base enumerável: $\beta = \{ (a, b) \subset \R \: | \: a, b \in \mathbb{Q} \}$.
    Sabemos que $\mathbb{Q}$ é enumerável e $\beta$ é base pelo \textbf{lema} anterior:
    dado $U \in \R$ e $x \in U$, temos que $x \in (a, b) \subset B(x, \eps) \subset U$.
\end{exemplo}

\begin{atencao}
    O produto de espaços que admitem base enumerável também admite base enumerável.
\end{atencao}

\begin{proposicao}
    Seja M espaço métrico. São equivalentes:
    \begin{enumerate}
        \item M admite base enumerável.
        \item $\exists D \subset M$ enumerável e denso (M é separavél).
    \end{enumerate}
\end{proposicao}

\begin{prova}
    $(1) \Rightarrow (2)$: \\
    Seja $\beta = \{ \beta_k\}$ base enumerável. 
    Para cada $k \in \mathbb{N}$ escolhemos $x_k \in \beta_k$,
    e então defina $D = \{x_k \: | \: k \in \mathbb{N}\}$. 
    D é natualmente enumerável pela construção, então precisamos verificar ainda que D é denso.
    Para ver que D é denso, basta ver que dado U aberto, vale $U \cap D \neq \emptyset$.
    Dado U, $\exists \: k$ tal que $\beta_k \neq \emptyset$ e $\beta_k \subset U$.
    Logo $x_k \in D \cap U$, e portanto D é denso.\\
    $(2) \Rightarrow (1)$: \\
    Seja D denso e enumerável e considere
    \[\beta = \{B(y, r) \: | \: y \in D, r \in \mathbb{Q}^+\}\]
    Temos que $\beta$ é enumerável e é base pelo $\textbf{lema}$:\\
    Seja U aberto e $x \in U$. Queremos B elemento da base tal que $x \in B \subset U$.
    Temos $\exists \: y \in D \cap B(x, \eps / 2)$, ou seja, $d(x, y) < \eps / 2$.
    Temos também, $\exists \: r \in \mathbb{Q}$ tal que $d(x, y) < r < \eps / 2$.
    Para este r, vale que $x \in B(y, r)$. 
    Assim $B(y, r) \subset B(x, \eps) \subset U$.
\end{prova}

\begin{definition}[\textbf{Base Local}]
    Dado M espaço métrico e $x \in M$, uma \textbf{base local de M em x} é $\beta = \{\beta_i\}_{i \in I}$
    de vizinhanças de x que verifica:
    $\forall \: U$ vizinhança de x, $\exists \: \beta_i \in \beta$ tal que $x \in B_i \subset U$.
\end{definition}

\begin{definition}[\textbf{Primeiro Axioma de Enumerabilidade}]
    Todo ponto admite base local enumerável.
\end{definition}

\begin{atencao}
    Todo espaço métrico satisfaz o Primeiro Axioma de Enumerabilidade. Basta tomar:
    \[\beta = \{ B(x, 1/k) \}\]
\end{atencao}

\section{Conexidade}

\begin{definition}[\textbf{Separação}]
    Seja M espaço métrico Uma \textbf{separação} de M é um par de subconjuntos de M $\{ A, B\}$ que verificam:
    \begin{itemize}
        \item $A \cup B = M$.
        \item $A \cap B = \emptyset$.
        \item A e B são abertos.
    \end{itemize}
    Chamamos $\{M, \emptyset\}$ de \textbf{separação trivial}.
\end{definition}

\begin{definition}[\textbf{Conexo}]
    M é \textbf{conexo} se a única separação que admite é a trivial. Se M não é conexo, dizemos que M é \textbf{desconexo}.
\end{definition}

\begin{atencao}
    Se $\{A, B\}$ é separação, então A e B são fechados.
\end{atencao}

\begin{proposicao}
    M é conexo se e somente se os únicos conjuntos abertos e fechados de M são M e $\emptyset$.
\end{proposicao}

\begin{prova}
    COMPLETAR!
\end{prova}

\begin{atencao}
    $X \subset M$ é conexo se X com a topologia relativa é conexo. 
\end{atencao}

\begin{proposicao}
    Se $f: M \rightarrow N$ é contínua e M é conexo, então $f(M)$ também é conexo.
\end{proposicao}

\begin{prova}
    Queremos ver que f(M) é conexo. Seja $\{A, B\}$ separação de f(M).
    $\exists \: A', B'$ abertos em N tais que 
    \[A' \cap f(M) = A\]
    \[B' \cap f(M) = B\]
    Como f é contínua, $f^{-1}(A')$ e $f^{-1}(B')$ são abertos de M.
    Logo
    \begin{align*}
        f^{-1}(A') &= f^{-1}(f(M) \cap A') \\
                   &= f^{1}(A) 
    \end{align*}
    e também $f^{-1}(B') = f^{-1}(B)$.\\
    Vamos mostrar que $\{f^{-1}(A), f^{-1}(B)\}$ é separação de M.
    \begin{itemize}
        \item $f^{-1}(A)$ e $f^{-1}(B)$ são abertos.
        \item $M = f^{-1}(f(M)) = f^{-1}(A \cup B) = f^{-1}(A) \cup f^{-1}(B)$.
        \item $f^{-1}(A) \cap f^{-1}(B) = f^{-1}(A \cap B) = \emptyset$.
    \end{itemize}
    Como M é conexo, temos duas opções: $f^{-1}(A) = \emptyset$ ou $f^{-1}(B) = \emptyset$.
    Se $f^{-1}(A) = \emptyset$, como $A \subset Im(f)$, segue que $A = \emptyset$ e então $B = f(M)$.
    Se $f^{-1}(B) = \emptyset$, analogo ao caso anterior temos, $B = \emptyset$ e $A = f(M)$.\\
    Assim $\{A, B\}$ é a separação trivial de f(M).
\end{prova}

\begin{corolario}
    Seja $f: M \rightarrow N$ homeomorfismo. Então, M é conexo se e somente se N é conexo.
\end{corolario}

\begin{exemplo}
    \
    \begin{itemize}
        \item $M = \{x\}$ é conexo.
        \item $[0, 1) \cup (1, 2]$ não é conexo. Basta perceber que $\{[0, 1), (1, 2]\}$ é separação, já que
            $[0, 1) = (-1, 1) \cap M$ e $(1, 2] = (1, 3) \cap M$ são abertos. 
    \end{itemize}
\end{exemplo}

\begin{theorem}
    Se $I \subset \R$ é um intervalo, então I é conexo. Mais ainda, os únicos conexos de $\R$ são intervalos ou conjuntos com um único ponto.
\end{theorem}

\begin{prova}
    Tomemos I intervalo de $\R$ e suponhamos que admite uma separação não trivial $\{A, B\}$.\\
    $\exists a \in A, b \in B$.
    Podemos supor que $a < b$. 
    Como I é um intervalo, vale $[a, b] \subset I$.
    Defina $C = \{x \in [a, b] \: | \: [a, x] \subset A\}$. 
    Vamos mostrar que $C \subset A$.\\
    Primeiro vamos mostrar que $C \neq \emptyset$. $a \in A$, como A é aberto em I.
    $\Rightarrow A \cap [a,b]$ é aberto em $[a,b]$.\\
    $\Rightarrow \: \exists \eps > 0 \: (a-\eps, a+\eps) \cap [a, b] \subset A$.\\
    Como $C \neq \emptyset$ e é limitado pois $C \subset [a,b]$. Existe $c = \sup(C)$.
    Como $\{A, B\}$ é separação, então $c \in A$ ou $c \in B$.
    \textbf{Afirmação:} $c \notin B$.\\
    Se $c \in B$, como B é aberto, $\exists \eps > 0$ tal que $(c-\eps, c+\eps) \subset B$.
    Logo $c-\eps/2$ é uma cota superior de C menor que c. Absurdo, já que c é $\sup(C)$. Então $c \in A$.
    \textbf{Afirmação:} $\sup(C) = b$
    Se $c < b$, como A é aberto, $\exists \eps > 0$ tal que $(c-\eps, c+\eps) \subset A$.
    Como $c = \sup(C)$, $\exists x \in (c-\eps/2, c]$ tal que\\
    $x \in C \: \Rightarrow \: [a,x] \subset A \: \Rightarrow \: (c-\eps, c + \eps/2] \subset A$.
    Logo a união $[a, c+\eps/2]$ está contido em A, logo $c+\eps/2 \in C$, absurdo já que $c = \sup(C)$.
    Então $c = b$ e portanto $b \in A \cap B$ um absurdo. 
\end{prova}

\begin{exemplo}
    \
    \begin{itemize}
        \item $S^1$ é conexo
        \item $S^1$ não é homeomorfo a nenhum intervalo de $\R$.
    \end{itemize}
\end{exemplo}

\begin{corolario}[\textbf{Teorema do Valor Intermediario}]
    Se $f: [a, b] \rightarrow \R$ é contínua e $f(a) < f(b)$, então $\forall \: d \in [f(a), f(b)]$ $\exists \: c \in [a, b]$ tal que $f(c) = d$. 
\end{corolario}

\begin{prova}
    COMPLETAR !!
\end{prova}

\begin{lema}
    Se M é espaço métrico, $\{A, B\}$ separação de M e $X \subset M$ é conexo, então $X \subset A$ ou $X \subset B$. 
\end{lema}

\begin{prova}
    $\{A \cap X, B \cap X\}$ é separação de X. Então ou $A \cap X = X$ ou $B \cap X = X$. Logo ou $X \subset A$ ou $X \subset B$.
\end{prova}

\begin{proposicao}
    Se $X \subset Y \subset \overline{X} \subset M$ com X conexo, então Y é conexo.
\end{proposicao}

\begin{prova}
    Seja $\{A, B\}$ separação de Y. Pelo lema anterior, como X é conexo, então ou $X \subset A$ ou $X \subset B$.
    Suponhamos que $X \subset A$, como A é fechado em Y $\overline{X}^Y \subset \overline{A}^Y = A$.
    Mas $\overline{X}^Y = \overline{X}^M \cap Y$. Logo $Y = A$.
    $\Rightarrow \{A, B\}$ é separação trivial de Y.
\end{prova}

\begin{proposicao}
    Seja $\{X_i\}_{i\in I}$ uma família de subconjuntos conexos em M,
    tais que $\bigcap_{i \in I}X_i \neq \emptyset$.
    Então $\bigcup_{i \in I}$ é conexo. 
\end{proposicao}

\begin{prova}
    Seja $\{A, B\}$ separação de $X = \bigcup_{i\in I}X_i$.
    Tome $x_0 \in \bigcap_{i \in I} X_i$.
    Pela separação, cada $X_i \subset A$ ou $X_i \subset B$.
    Se $x_0 \in A$, então $X_i \subset A \quad \forall \: i$.
    Assim $A = X$ e $B = \emptyset$.
\end{prova}

\begin{proposicao}
    Se $M_1, M_2, \dots, M_n$ são espaços conexos, então
    $M = M_1 \times M_2 \times \dots \times M_n$ é conexo. 
\end{proposicao}

\begin{prova}
    Vamos provar por indução em n, quantidade de espaços conexos.\\
    Como caso base vamos provar para $n = 2$. 
    Sejam M e N conexos. Tome um ponto $(x, y')$ em $M \times N$.
    Temos que $\{x\} \times N$ é homeomorfo a N, logo é conexo.
    De maneira analoga temos que $M \times \{y'\}$ é conexo.
    Defina então $C_x = (\{x\} \times N) \cup (M \times \{y'\})$.
    $C_x$ é conexo pois é união de dois conexos com um ponto em comum, $(x, y')$.
    Agora $M \times N = \bigcup_{x \in M} C_x$, com cada $C_x$ conexo 
    e $\bigcap_{x \in M} C_x = M \times \{y'\}$.
    Então pela proposicao novamente, $M \times N$ é conexo.
    Seguindo pela indução temos:
    \[M_1 \times \dots \times M_n \times M_{n+1} = (M_1 \times \dots \times M_n) \times M_{n+1}\]
    Então pela hipotese de indução, é conexo.
\end{prova}

\begin{definition}[\textbf{Componentes Conexas}]
    Dado M espaço métrico, definimos $\sim$ em M, x $\sim$ y se existe 
    $C \subset M$ conexo tal que $x, y \in C$. 
    Chamamos as classes de N como \textbf{componentes conexas} de M.
\end{definition}

\begin{proposicao}
    M espaço métrico, $\{C_i\}_{i\in I}$ as componentes conexas de M. Então:
    \begin{enumerate}
        \item $M = \bigcup_{i \in I} C_i$.
        \item Se $i \neq j$ então $C_i \cap C_j = \emptyset$.
        \item Cada $C_i$ é conexa.
        \item Se $C \subset M$ é conexo, então existe $i \in I$ tal que $C \subset C_i$.
        \item Cada $C_i$ é fechada.
        \item M é conexo se e somente se $|I| = 1$.
    \end{enumerate}
\end{proposicao}

\begin{prova}
    \textbf{(1) e (2)} são imediatos, pois $\sim$ é uma relação de equivalencia e
    as classes formam uma partição do espaço.\\
    \textbf{(3)}: Fixemos $x \in C_i$, e para cada $y \in C_i$ seja $D_y$ conexo de M
    tal que $x, y \in D_y$.
    Então $C_i = \bigcup_{y \in C_i} D_y$ e $\{x\} \subset \bigcap_{y \in C_i} D_y$.
    Logo $C_i$ é união de conexos com um ponto em comum e portanto é conexo.\\
    \textbf{(4)}: Se $C \subset M$ é conexoe $x \in C$, seja $C_i$ tal que $x \in C_i$.
    $\forall \: y \in  C, y \sim x \Rightarrow y \in C_i \Rightarrow C \subset C_i$. \\
    \textbf{(5)}: $C_i$ conexo, então $\overline{C_i}$ é conexo.
    Mas pelo item (4) $\overline{C_i} \subset C_i$, portanto $C_i = \overline{C_i}$.\\
    \textbf{(6)}: Trivial.
\end{prova}

\begin{atencao}
    $C_i$ nem sempre é aberto.
    Precisamos de uma quantidade finita de compontes conexas para que cada uma delas seja aberta.
\end{atencao}

\begin{definition}[\textbf{Totalmente Disconexo}]
    Se $\forall i \in I$, $|C_i| = 1$, ou seja, $C_i$ é um único ponto, dizemos que 
    M é \textbf{totalmente disconexo}.
\end{definition}

\begin{exemplo}
    \
    \begin{itemize}
        \item $\mathbb{Q}$ é totalmente disconexo.
        \item Todo espaço discreto é totalmente disconexo.
    \end{itemize}
\end{exemplo}

\begin{atencao}
    Conexão é uma propriedade global.
     Ou seja, temos que entender todo o espaço para determinar se verifica a propriedade.
\end{atencao}

\begin{definition}[\textbf{Localmente Conexo}]
    M é \textbf{localmente conexo} se $\forall \: x, \forall \: U_x$ vizinhança de x $\exists V_x$
    vizinhança de x conexa com $V_x \subset U_x$.
\end{definition}

\begin{exemplo}
    \
    \begin{itemize}
        \item $\R$ e $\R^n$ são localmente conexos.
        \item $M = [0, 1) \cup (1, 2]$ é localmente conexo, mas não é conexo.
    \end{itemize}
\end{exemplo}

\begin{proposicao}
    Seja M espaço métrico localmente conexo, então toda componente conexa é aberta.
\end{proposicao}

\begin{prova}
    Seja C componente conexa e $x \in C$.
    $\exists \: V_x$ aberto conexo que contém x. 
    Então $V_x \subset C$ e portanto x é ponto de interior de C.
    Logo C é aberto. 
\end{prova}

\begin{theorem}
    Seja M conexo, $\forall \: x \in M$ $\exists \: \eps > 0$ tal que $B(x, \eps)$ é separavél.
    Então M é separavél.
\end{theorem}

\begin{prova}
    Se considerarmos $A \subset M$ tal que se provarmos que $A = M$ então concluimos o teorema.
    Para provarmos que $A = M$, mostraremos que 
    \begin{enumerate}
        \item $A \neq \emptyset$
        \item A é aberto
        \item A é fechado
    \end{enumerate}
    Assim como M é conexo, ganhamos que $A = M$.
\end{prova}

\end{document}