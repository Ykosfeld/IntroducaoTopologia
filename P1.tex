\documentclass{article}
\usepackage{amsthm}
\usepackage{amsmath}
\usepackage{amsfonts}

\newtheorem*{definition}{Definição}
\newtheorem*{theorem}{Teorema}
\newtheorem*{exemple}{Exemplo}
\newtheorem*{atencao}{Atenção}
\newtheorem*{proposicao}{Proposição}
\usepackage{mathtools}
\usepackage{xfrac}
\DeclarePairedDelimiter\abs{\lvert}{\rvert}%
\DeclareMathOperator{\Ima}{Im}

\setlength{\textwidth}{450pt}
\setlength{\marginparwidth}{0pt}
\setlength{\marginparsep}{0pt}
\usepackage[left=2.5cm, right=2.5cm]{geometry}

\title{Definições e Propriedades P1 \\ \large Introdução a Topologia}
\author{Yuri Kosfeld}
\date{Abril 2025}

\begin{document}

\maketitle

\begin{definition}[\textbf{Distância}]
    Seja M um conjunto. Uma \textbf{distância} em M é uma função $d: M \times M \rightarrow \left[0, \infty \right)$
    que satisfaz as seguintes propriedades: 
    \begin{enumerate}
        \item $d(x,x) = 0$ para todo $x \in M$.
        \item $d(x,y) > 0$ para todos $x, y \in M$ com $x \neq y$.
        \item $d(x, y) = d(y, x)$ para todos $x,y \in M$.
        \item $d(x,z) \leq d(x, y) + d(y, z)$ para todos $x, y, z \in M$.
    \end{enumerate}
\end{definition}

\begin{definition}[\textbf{Bola Aberta}]
    Sejam (M, d) um espaço métrico, $x\in M$ e $\varepsilon > 0$. Definimos a \textbf{bola aberta}
    centrada em x e de raio $\varepsilon$ como:
    \[ B(x, \varepsilon) = \{ y \in M \: | \: d(x, y) < \varepsilon \} \]
\end{definition}

\begin{definition}[\textbf{Bola da distância induzida}]
    \[B_A(x, \varepsilon) = \{y \in A \: | \: d_A (x,y) < \varepsilon \}\]
    \[B_A(x, \varepsilon) = B(x, \varepsilon) \cap A\]
\end{definition}

\begin{definition}[\textbf{Conjunto Aberto}]
    $U \subset M$ é \textbf{conjunto aberto} se $\forall \: x \in U$ $\exists \: \varepsilon > 0$ tal que $B(x, \varepsilon) \subset U$.
\end{definition}

\begin{definition}[\textbf{Vizinhança}]
    U aberto tal que $x \in U$.
\end{definition}

\begin{definition}[\textbf{Topologia}]
    Seja (M, d) espaço metrico, dizemos que \textbf{topologia} é a familia de todos os subconjuntos abertos de M.
    \[ \mathcal{T} = \{ U \subset M \: | \: \text{U é aberto}\} \]
\end{definition}

\begin{definition}[\textbf{Aberto EM}]
    Dizemos que U é \textbf{aberto em A} $\Leftrightarrow$ $\exists \: V$ aberto em M tal que $U = V \cap A$.
\end{definition}

\begin{definition}[\textbf{Ponto de Interior}]
    Dizemos que $x \in A$ é um \textbf{ponto de interior} de A se existir U vizinhança de x tal que $U \subset A$.
    Temos então int(A).
\end{definition}

\begin{proposicao}
    Valem:
    \begin{enumerate}
        \item $int(A) \subset A$.
        \item $int(int(A)) = int(A)$
        \item $int(A \cap B) = int(A) \cap int(B)$
        \item $int(A \cup B) \subset int(A) \cup int(B)$
    \end{enumerate}
\end{proposicao}

\begin{definition}[\textbf{Ponto Limite}]
    Seja (M, d) espaço metrico e $A \subset M$. Dizemos que $x \in M$ é \textbf{ponto limite} de A, se $\forall \: \varepsilon > 0$ temos que
    $B(x, \varepsilon) \cap A \setminus {x} \neq \emptyset $. Conjunto de todos os pontos limites de A: A'.
\end{definition}

\begin{definition}[\textbf{Ponto Isolado}]
    Dizemos que x é \textbf{ponto isolado} de A, se $x \in A \setminus A'$, ou seja, $\exists \: \varepsilon > 0$ tal que 
    $B(x, \varepsilon) \cap A \setminus {x} = \emptyset $
\end{definition}

\begin{definition}[\textbf{Conjunto Fechado}]
    Dizemos que $F \subset M$ é um \textbf{conjunto fechado} se $F' \subset F$.
\end{definition}

\begin{proposicao}
    (M, d) espaço metrico e $A \subset M$. A é aberto $\Leftrightarrow$ $A^c$ é fechado.
\end{proposicao}

\begin{definition}[\textbf{Fecho}]
    Definimos o \textbf{fecho} de um conjunto A como $\overline{A} = A \cup A'$.
\end{definition}

\begin{proposicao} Valem:
    \begin{enumerate}
        \item  $x \in \overline{A} \Leftrightarrow \forall \: \varepsilon > 0 \: B(x, \varepsilon) \cap A \neq \emptyset$
        \item $ \text{F é fechado} \Leftrightarrow \overline{F} = F$
    \end{enumerate}
\end{proposicao}

\begin{definition}[\textbf{Denso}]
    (M, d) espaço metrico, $A \subset M$. Dizemos que A é \textbf{denso} se $\overline{A} = M$.
\end{definition}

\begin{proposicao}
    A é denso $\Leftrightarrow$ para todo U aberto de M, $U \cap A \neq \emptyset$.
\end{proposicao}

\begin{definition}[\textbf{Conjunto Perfeito}]
    (M, d) espaço metrico, $A \subset M$. A é \textbf{perfeito} se $A = A'$. Todo conjunto perfeito é fechado.
\end{definition}

\begin{definition}
    Dizemos que um conjunto $A \subset M$ é \textbf{discreto} se para todo $x \in A$, existe $\varepsilon > 0$ tal que $B(x, \varepsilon) \cap A = \{x\}$.
\end{definition}

\begin{definition}[\textbf{Fronteira}]
    (M, d) espaço metrico, $A \subset M$. A \textbf{fronteira} de A é definida como: $\partial A = \overline{A} \cap \overline{A^c} $
\end{definition}

\begin{proposicao}
    Valem:
    \begin{enumerate}
        \item $int(A)$ e $\partial A $ são disjuntos.
        \item $\overline{A} = int(A) \cup \partial A$
        \item $\partial A = \emptyset \: \Leftrightarrow$ A é aberto e fechado ao mesmo tempo.
    \end{enumerate}
\end{proposicao}

\begin{definition}[\textbf{Sequencia Convergente}]
    Seja $\{x_k\}_{k \in \mathbb{N}}$ uma sequencia em M. ${x_k}$ é \textbf{convergente} se:
    \[\forall \: \varepsilon > 0 \: \exists \: k_0 \in \mathbb{N} \quad \text{tal que} \quad x_k \in B(x, \varepsilon) \: \forall \: k \geq k_0\] 
\end{definition}

\begin{proposicao}
    Valem: 
    \begin{enumerate}
        \item A convergencia é unica.
        \item M um conjunto e $d_1$ e $d_2$ duas distancias topologicamente equivalentes. Então $x_k$ converge a x por $d_1$
            se e somente se $x_k$ converge a x por $d_2$.
        \item M espaço metrico e $A \subset M$. Então, $x \in \overline{A}$ se e somente se existe uma sequencia $\{a_k\}_{k \in \mathbb{N}}$
            de pontos em A que convergem a x.
        \item Sejam $M_1, \dots, M_n$ espaços metricos, e defina $M = M_1 \times \dots \times M_n$ espaço metrico.
            Uma sequencia $x_k$ em M é convergente se e somente se cada sequencia coordenada for convergente.
    \end{enumerate}
\end{proposicao}

\begin{definition}[\textbf{Sequencia de Cauchy}]
    Dizemos que ${x_k}_{k \in \mathbb{N}}$ é uma \textbf{sequencia de cauchy} se:
     \[\forall \: \varepsilon > 0 \: \exists \: k_0 \in \mathbb{N} \quad \text{tal que se} \quad k, l \geq k_0 \: \text{então} \: d(x_k, x_l) < \varepsilon\]
\end{definition}

\begin{definition}[\textbf{espaço Completo}]
    Um espaço metrico M é \textbf{completo} se toda sequencia de Cauchy em M converge para um ponto de M.
\end{definition}

\begin{definition}[\textbf{Ponto de Aderencia}]
    Dizemos que x é um \textbf{ponto de aderencia} de $x_k$ se existe uma sub de $x_k$ que converge a x.
\end{definition}

\begin{proposicao}
    Valem:
    \begin{enumerate}
        \item Se $x_k$ é convergente, então é de Cauchy.
        \item Se $x_k$ é de Cauchy e possui uma subsequencia convergente, então $x_k$ é convergente.
        \item Se $x_k \subset \mathbb{R}$ é monotona e limitada, então $x_k$ é convergente.
        \item Se $x_k \subset \mathbb{R}$ é limitada, então $x_k$ possui uma sub convergente.
        \item $\mathbb{R}$ é completo.
    \end{enumerate}
\end{proposicao}

\begin{definition}[\textbf{Eventualmente Constante}]
    Dizemos que $x_k$ é \textbf{eventualmente constante} se existe $k_0$ tal que para todo $k, m \geq k_0$ temos $x_k = x_m$.
\end{definition}

\begin{definition}[\textbf{Isometria}]
    Dizemos que $f: M \rightarrow N$ é uma \textbf{isometria} se é bijetiva e:
    \[\forall \: x, y \in M \quad d_N(f(x), f(y)) = d_M(x, y)\]
\end{definition}

\begin{definition}[\textbf{Homeomorfismo}]
    Dizemos que f é um \textbf{homeomorfismo} se é bijetiva e:
    \[U \in T_M \quad \text{se e somente se} \quad f(U) \in T_N\]
\end{definition}

\begin{definition}[\textbf{Continua}]
    Uma função $f: M \rightarrow N$ é dita \textbf{continua}, se para todo aberto $U \subset N$, a pré-imagem $f^{-1}(U)$ é aberto em M.
\end{definition}

\begin{proposicao}
    Valem as equivalencias:
    \begin{enumerate}
        \item f é continua
        \item Para todo F fechado em N, $f^{-1}(F)$ é fechado em M.
        \item Para todo $x \in M$ e para todo $varepsilon >0$, existe $\varDelta > 0 $ tal que $f(B(x, \varDelta)) \subset B(f(x), \varepsilon)$.
        \item Para toda sequencia $x_k$ em M e x em M, se $x_k$ converge a x então $f(x_k)$ converge a $f(x)$.
        \item Para todo A em M, $f(\overline{A}) \subset \overline{f(A)}$.
    \end{enumerate}
\end{proposicao}

\end{document}