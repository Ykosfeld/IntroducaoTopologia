\documentclass{article}
\usepackage{amsthm}
\usepackage{amsmath}
\usepackage{amsfonts}

\newtheorem*{definition}{Definição}
\newtheorem*{theorem}{Teorema}
\newtheorem*{exemple}{Exemplo}
\newtheorem*{atencao}{Atenção}
\newtheorem*{proposicao}{Proposição}
\usepackage{mathtools}
\usepackage{xfrac}
\DeclarePairedDelimiter\abs{\lvert}{\rvert}%
\DeclareMathOperator{\Ima}{Im}

\setlength{\textwidth}{450pt}
\setlength{\marginparwidth}{0pt}
\setlength{\marginparsep}{0pt}
\usepackage[left=2.5cm, right=2.5cm]{geometry}

\title{Definições e Propriedades P1 \\ \large Introdução a Topologia}
\author{Yuri Kosfeld}
\date{Abril 2025}

\begin{document}

\maketitle

\begin{definition}[\textbf{Distância}]
    Seja M um conjunto. Uma \textbf{distância} em M é uma função $d: M \times M \rightarrow \left[0, \infty \right)$
    que satisfaz as seguintes propriedades: 
    \begin{enumerate}
        \item $d(x,x) = 0$ para todo $x \in M$.
        \item $d(x,y) > 0$ para todos $x, y \in M$ com $x \neq y$.
        \item $d(x, y) = d(y, x)$ para todos $x,y \in M$.
        \item $d(x,z) \leq d(x, y) + d(y, z)$ para todos $x, y, z \in M$.
    \end{enumerate}
\end{definition}

\begin{definition}[\textbf{Bola Aberta}]
    Sejam (M, d) um espaço métrico, $x\in M$ e $\varepsilon > 0$. Definimos a \textbf{bola aberta}
    centrada em x e de raio $\varepsilon$ como:
    \[ B(x, \varepsilon) = \{ y \in M \: | \: d(x, y) < \varepsilon \} \]
\end{definition}

\begin{definition}[\textbf{Bola da distância induzida}]
    \[B_A(x, \varepsilon) = \{y \in A \: | \: d_A (x,y) < \varepsilon \}\]
    \[B_A(x, \varepsilon) = B(x, \varepsilon) \cap A\]
\end{definition}

\begin{definition}[\textbf{Conjunto Aberto}]
    $U \subset M$ é \textbf{conjunto aberto} se $\forall \: x \in U$ $\exists \: \varepsilon > 0$ tal que $B(x, \varepsilon) \subset U$.
\end{definition}

\begin{definition}[\textbf{Vizinhança}]
    U aberto tal que $x \in U$.
\end{definition}

\begin{definition}[\textbf{Topologia}]
    Seja (M, d) espaço metrico, dizemos que \textbf{topologia} é a familia de todos os subconjuntos abertos de M.
    \[ \mathcal{T} = \{ U \subset M \: | \: \text{U é aberto}\} \]
\end{definition}

\begin{definition}[\textbf{Aberto EM}]
    Dizemos que U é \textbf{aberto em A} $\Leftrightarrow$ $\exists \: V$ aberto em M tal que $U = V \cap A$.
\end{definition}

\begin{definition}[\textbf{Ponto de Interior}]
    Dizemos que $x \in A$ é um \textbf{ponto de interior} de A se existir U vizinhança de x tal que $U \subset A$.
    Temos então int(A).
\end{definition}

\begin{proposicao}
    Valem:
    \begin{enumerate}
        \item $int(A) \subset A$.
        \item $int(int(A)) = int(A)$
        \item $int(A \cap B) = int(A) \cap int(B)$
        \item $int(A \cup B) \subset int(A) \cup int(B)$
    \end{enumerate}
\end{proposicao}

\begin{definition}[\textbf{Ponto Limite}]
    Seja (M, d) espaço metrico e $A \subset M$. Dizemos que $x \in M$ é \textbf{ponto limite} de A, se $\forall \: \varepsilon > 0$ temos que
    $B(x, \varepsilon) \cap A \setminus {x} \neq \emptyset $. Conjunto de todos os pontos limites de A: A'.
\end{definition}

\begin{definition}[\textbf{Ponto Isolado}]
    Dizemos que x é \textbf{ponto isolado} de A, se $x \in A \setminus A'$, ou seja, $\exists \: \varepsilon > 0$ tal que 
    $B(x, \varepsilon) \cap A \setminus {x} = \emptyset $
\end{definition}

\begin{definition}[\textbf{Conjunto Fechado}]
    Dizemos que $F \subset M$ é um \textbf{conjunto fechado} se $F' \subset F$.
\end{definition}

\begin{proposicao}
    (M, d) espaço metrico e $A \subset M$. A é aberto $\Leftrightarrow$ $A^c$ é fechado.
\end{proposicao}

\begin{definition}[\textbf{Fecho}]
    Definimos o \textbf{fecho} de um conjunto A como $\overline{A} = A \cup A'$.
\end{definition}

\begin{proposicao} Valem:
    \begin{enumerate}
        \item  $x \in \overline{A} \Leftrightarrow \forall \: \varepsilon > 0 \: B(x, \varepsilon) \cap A \neq \emptyset$
        \item $ \text{F é fechado} \Leftrightarrow \overline{F} = F$
    \end{enumerate}
\end{proposicao}

\begin{definition}[\textbf{Denso}]
    (M, d) espaço metrico, $A \subset M$. Dizemos que A é \textbf{denso} se $\overline{A} = M$.
\end{definition}

\end{document}